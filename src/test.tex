
\documentclass[]{deedy-resume-openfont}
\usepackage{fancyhdr}
\usepackage{glossaries}

\pagestyle{fancy}
\fancyhf{}

\begin{document}

%%%%%%%%%%%%%%%%%%%%%%%%%%%%%%%%%%%%%%
%
%     LAST UPDATED DATE
%
%%%%%%%%%%%%%%%%%%%%%%%%%%%%%%%%%%%%%%
    \lastupdated

%%%%%%%%%%%%%%%%%%%%%%%%%%%%%%%%%%%%%%
%
%     TITLE NAME
%
%%%%%%%%%%%%%%%%%%%%%%%%%%%%%%%%%%%%%%
    \namesection{Arpit}{Sharma}{
        \href{mailto:sharmaarpit711b@gmail.com}{sharmaarpit711b@gmail.com} | 7996900298 \\
        \href{http://linkedin.com/in/arpit-sharma-3aba1097}{www.linkedin.com/in/arpit-sharma-3aba1097}
    }

%%%%%%%%%%%%%%%%%%%%%%%%%%%%%%%%%%%%%%
%
%     COLUMN ONE
%
%%%%%%%%%%%%%%%%%%%%%%%%%%%%%%%%%%%%%%

    \begin{minipage}[t]{0.33\textwidth}

%%%%%%%%%%%%%%%%%%%%%%%%%%%%%%%%%%%%%%
%     EDUCATION
%%%%%%%%%%%%%%%%%%%%%%%%%%%%%%%%%%%%%%

        \section{Education}

        \subsection{IIIT-Allahabad}
        \descript{M.Tech in Information Technology}
        \location{Aug 20 | Allahabad, India}
        \location{Cum. GPA: 8.5 / 10.0}
        \sectionsep


% \subsection{La Martiniere for Boys}
% \location{Grad. May 2011|  Kolkata, India}
% \sectionsep

%%%%%%%%%%%%%%%%%%%%%%%%%%%%%%%%%%%%%%
%     LINKS
%%%%%%%%%%%%%%%%%%%%%%%%%%%%%%%%%%%%%%

        \section{Links}
        Interviewbit:// \href{https://www.interviewbit.com/profile/arpit711}{\bf arpit711} \\
        Leetcode:// \href{https://leetcode.com/arpit711/}{\bf arpit711} \\
% Github:// \href{https://github.com/deedydas}{\bf deedydas} \\
        LinkedIn://  \href{http://linkedin.com/in/arpit-sharma-3aba1097}{\bf ArpitSharma} \\
% YouTube://  \href{https://www.youtube.com/user/DeedyDash007}{\bf DeedyDash007} \\
% Twitter://  \href{https://twitter.com/debarghya_das}{\bf @debarghya\_das} \\
% Quora://  \href{https://www.quora.com/Debarghya-Das}{\bf Debarghya-Das}

%%%%%%%%%%%%%%%%%%%%%%%%%%%%%%%%%%%%%%
%     COURSEWORK
%%%%%%%%%%%%%%%%%%%%%%%%%%%%%%%%%%%%%%

        \section{Coursework}
        \subsection{Graduate}
        Advance Data Structures \\
        Design and Analysis of Algorithms \\
        Principles of Programming Languages \\
        Software Engineering Designs\\
        Machine Learning \\

        \sectionsep

        \subsection{Undergraduate}
        Data Structures and Algorithms \\
        Operating Systems \\
        OOPS Java \\
        Computer Networks\\
% {\footnotesize \textit{\textbf{(Research Asst. \& Teaching Asst 2x) }}} \\
% Unix Tools and Scripting \\

%%%%%%%%%%%%%%%%%%%%%%%%%%%%%%%%%%%%%%
%     SKILLS
%%%%%%%%%%%%%%%%%%%%%%%%%%%%%%%%%%%%%%

        \section{Skills}
        \subsection{Programming}
        Java \textbullet{} C/C++ \textbullet{} Python \textbullet{} Spring/Spring Boot\\
        \location{Over 1000 lines:}
        Java \textbullet{} C++ \textbullet{} JUnit \textbullet{} Mockito \textbullet{} AWS

        \sectionsep
        \subsection{Tech Stack}
        AWS \textbullet{} Kubernetes \textbullet{} Docker \textbullet{} Microservices \\
        Argo Workflows \textbullet{} Kafka \textbullet{} Redis \\ \textbullet{} Node.js \textbullet{} Flask  \textbullet{} Javascript \textbullet{} HTML/CSS \\
        PySpark \textbullet{} Amazon SQS/SNS \textbullet{} Git \textbullet{} GitLab CI/CD

        \section{Achievements}
        \textbullet{} Outperformed 97.3 percent of applicants (All India) in the Graduate Aptitude Test in Engineering (GATE) - CS (2016).
        \sectionsep

%%%%%%%%%%%%%%%%%%%%%%%%%%%%%%%%%%%%%%
%
%     COLUMN TWO
%
%%%%%%%%%%%%%%%%%%%%%%%%%%%%%%%%%%%%%%

    \end{minipage}
    \hfill
    \begin{minipage}[t]{0.66\textwidth}

%%%%%%%%%%%%%%%%%%%%%%%%%%%%%%%%%%%%%%
%     EXPERIENCE
%%%%%%%%%%%%%%%%%%%%%%%%%%%%%%%%%%%%%%

        \section{Experience}

        \runsubsection{Arcesium India Pvt Ltd.}
        \descript{| Software Engineer Backend Developer}
        \location{October 2021 - Present | Gurugram, India}
        \vspace{\topsep} % Hacky fix for awkward extra vertical space
        \begin{tightemize}
            % \newline

            \item Successfully built sensors and events using AWS SNS and SQS.
            \item Built async executing workflows in Kubernetes using Argo Workflows.
            \item Optimized Kubernetes resources to prevent pod termination by introducing executors inside drivers.
            \item Created unit and integration tests using JUnit and Mockito.
            \sectionsep

            \descript{Back End Application in Java and Python}
            \item \textbf{Updated Backend CRUD REST APIs:} Enhanced and maintained RESTful APIs by adding new endpoints, optimizing existing functionalities, and ensuring compliance with the latest standards. Implemented versioning to support backward compatibility and improve client integration.
            \item \textbf{Optimized Performance:} Improved API response times by 30\% through refactoring code, optimizing database queries, and implementing caching strategies.
            \item \textbf{Testing and Documentation:} Developed unit and integration tests using JUnit and Postman. Documented API endpoints and usage in Swagger, enhancing team collaboration and reducing integration errors.
            \newline
            \newline
            \descript{Microservices in Java and Python}
            \item \textbf{Microservices and ETL Pipelines:} Designed and implemented microservices architecture for scalable data processing and integration. Developed ETL (Extract, Transform, Load) pipelines to handle data ingestion, transformation, and loading into data warehouses.
            \item \textbf{Integration and Communication:} Used REST and message queues (e.g. AWS SQS/SNS, Kafka) for inter-service communication. Ensured data consistency and reliability through robust error handling and retry mechanisms.
            \item \textbf{Performance and Monitoring:} Implemented performance monitoring using Datadog. Analyzed and optimized microservices for better scalability and resilience.
            \item \textbf{Continuous Integration/Continuous Deployment (CI/CD):} Automated deployment pipelines using GitLab CI/CD, enabling faster and more reliable deployments with integrated testing and continuous delivery workflows.
        \end{tightemize}
        \sectionsep

        \runsubsection{Mediatek}
        \descript{| Engineer R\&D}
        \location{Aug 2020 - Sept 2021 | Noida, India}
        \vspace{\topsep} % Hacky fix for awkward extra vertical space
        \begin{tightemize}
            \item Worked on Android Linux kernel for MTK proprietary modules CMDQ/GCE.
            \item Developed microservices using Spring Boot with Gradle build tools and annotations.
            \item Implemented multiple design patterns like Factory and Publisher-Subscriber.
        \end{tightemize}
        \sectionsep

%%%%%%%%%%%%%%%%%%%%%%%%%%%%%%%%%%%%%%
%     RESEARCH
%%%%%%%%%%%%%%%%%%%%%%%%%%%%%%%%%%%%%%

%   \section{Research}
%   \runsubsection{Indian Institute of Information Technology Allahabad (IIIT-A)}
%   \descript{| M.Tech (IT)}
% % \location{Jan 2020 – Aug 2020 | Allahabad, India}
%   \runsubsection{A Comparative Study for Ad-Click Prediction}
%   \descript{| Python}
%   \location{Jan 2020 – Aug 2020 | Allahabad, India}
%   \begin{tightemize}
%    \item Predicted whether a user will click the advertisement shown to them based on past click data and likes.
%    \item Compared multiple Machine Learning models for CTR (Click Through Rate) prediction.
%   \end{tightemize}
%   \sectionsep

%%%%%%%%%%%%%%%%%%%%%%%%%%%%%%%%%%%%%%
%     AWARDS
%%%%%%%%%%%%%%%%%%%%%%%%%%%%%%%%%%%%%%

% \section{Awards}
% \begin{tabular}{rll}
%  Tournament  \\
% 2012     & 2\textsuperscript{nd}/150 & CS 3110 Biannual Intra-Class Bot Tournament \\
% 2011     & National & Indian National Mathematics Olympiad (INMO) Finalist \\
% \end{tabular}
% \sectionsep

%%%%%%%%%%%%%%%%%%%%%%%%%%%%%%%%%%%%%%
%     PUBLICATIONS
%%%%%%%%%%%%%%%%%%%%%%%%%%%%%%%%%%%%%%

        \nocite{*}

    \end{minipage}
\end{document}  \documentclass[]{article}
